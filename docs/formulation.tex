
\section{Mathematical Formulation} \label{sec:formulation}

We focus on problems ...

We consider systems which my be expressed in semi-discrete form as
\begin{equation}
%r(\dot{\state},\state,t;\params)=0
\dot{\state} = \velocity(\state,t;\params)
\end{equation}
where $\state\in\RR{\FomDim}$, 
$t\in [0,T] $, 
and $\mu\in\mathcal{D}$.
%and $r: \RR{\FomDim}\times\RR{\FomDim}\times[0,T]\times\mathcal{D}\to\RR{\FomDim}$.

We assume that we are able evaluate $\state(t;\params)$ for multiple ``training" parameter instances $\params_i\in\mathcal{D}$ and to save off a set of snapshots of the state $\snapshots=\left[ \state(t;\params_1) \dots \state(t;\params_{n_t}) \right] \in \RR{N\times n_t}$.

\subsection{Galerkin Model Reduction}

In the Galerkin method, we seek to approximate $\state$ as 
\begin{equation}
\approxState\approx\basis\RedState
\end{equation}
with %$\refstate\in\RR{\FomDim}$, 
$\RedState\in\RR{\RomDim}$, and $\basis\in\RR{\FomDim\times\RomDim}$ and $\FomDim\gg\RomDim$.
The Galerkin reduced order model (ROM) is constructed by projecting the governing equation onto the basis $\basis$
\begin{equation}
\dot{\RedState} = (A\basis)^\dagger A\velocity(\basis\RedState,t;\params)
\end{equation}
where $\dagger$ denotes the Moore-Penrose pseudoinverse and $A\in\RR{\HRDim\times\FomDim}$ with $\HRDim<\FomDim$ is a weighting matrix to enable hyperreduction.

Traditionally, $\basis$ is created via POD.
We assemble the snapshot matrix $\snapshots$ and compute the (thin) singular value decomposition
\begin{equation}
U,\Sigma, V = svd(\snapshots).
\end{equation}
The left singular vectors $U\in\RR{N\times min(N,n_t)}$ form a hierarchical basis for the solution $\state$.
We collect the first $K\leq n_t$, $K\ll N$ columns of $U$ into the basis matrix $\basis$.

\subsection{OD Galerkin} \label{sec:od_galerkin}

In the OD Galerkin method, the state is decomposed into $x_0, x_1, \dots, \state_{n-1}$, $\state_i\in\RR{\tileDim}$ with
\begin{equation}
\sum_{i=0}^{n-1} \tileDim = \FomDim
\end{equation}
and 
\begin{equation}
x=[x_0^T, x_1^T, \dots, x_{n-1}^T]^T.
\end{equation}
This decomposition represents the partitioning of the domain into a set of subdomains via e.g. recursive coordinate bisection (RCB).

We can compute a local basis $\basis_i\in\RR{\tileDim\times\RomDim_i}$ for each local state $\state_i$ using the method of snapshots described above.
Note that the local basis needent be of the same dimension for each subdomain.
Within each subdomain, we can apply Galerkin projection to yield
\begin{equation}
\label{eq:local_rom}
\dot{\RedState}_i = (A_i\basis_i)^\dagger A_i\velocity(\state_{ref,i} + \basis_i\RedState_i,t;\params)
\end{equation}
Assuming that the host code is capable of parallel evaluation, the existing machinery may be leveraged to provide each local ROM (Eq. \ref{eq:local_rom}) with the appropriate boundary conditions at the partition interfaces (either Schur's complement or Schwarz alternating methods depending on the host code architecture).

This is equivalent to defining a global ROM with a block diagonal basis matrix given by
\begin{equation}
\label{eq:block_basis}
\Phi = 
\begin{bmatrix}
\Phi_0 & 0 & \dots & 0 \\
0      & \Phi_1 &  &  \\
\vdots & & \ddots & \vdots \\
0 & & \dots & \Phi_{n-1}
\end{bmatrix},
\end{equation}
with $\Phi\in\RR{\FomDim\times\RomDim}$ and $\RomDim=\sum \RomDim_i$.
The use of spatially local bases within each partition is observed to provide the resultant OD Galerkin ROM with enhanced stability realtive to a global ROM.

\subsection{GFEM}

The generalized finite element method (GFEM) is an approach to approximating functions with 3 main elements: patches, a partition of unity, and local approximation spaces \cite{shilt2021high}.  GFEM has been shown to provide stability benefits for a range of applications including Stokes flow \cite{SHILT2020113165}, advection-diffusion \cite{SHILT2021113889}, and unsteady Burgers' \cite{shilt2021high}.

\JT{these definitions are just copied from \cite{shilt2021high} with the notation switched around a little}

\subsubsection{Patches} \label{sec:patches}

Define an open covering of the spatial domain $\Omega$ such that
\begin{equation}
\Omega\subset\bigcup_{i=0}^{n-1} \omega_i.
\end{equation}
Any point $r\in\Omega$ may belong to at most $m\le n$ elements of $\lbrace\omega_i\rbrace_{i=0}^{n-1}$.
For a discretized domain, $\omega_i$ commonly consists of the union of all finite elements sharing the node $i$ of the mesh.

\subsubsection{Partition of Unity}
Let $\lbrace\psi_i\rbrace_{i=0}^{n-1}$ be piecewise $C^0$ functions defined on $\Omega$ satsifying
\begin{equation}
\sum_{i=0}^{n-1} \psi(r) = 1, \forall r\in\Omega
\end{equation}
then $\lbrace\psi_i\rbrace_{i=0}^{n-1}$ forms a partition of unity with respect to $\lbrace\omega_i\rbrace_{i=0}^{n-1}$.

\subsubsection{Local Approximation Spaces}

In GFEM, a function $u(r)$ is approximated as 
\begin{equation}
\label{eq:gfem}
u(r)\approx \sum_i \psi_i(r) u_i + \sum_\alpha \sum_j \psi_j(r)\xi_\alpha(r)a_{j\alpha}
\end{equation}
with $r\in\Omega$ \cite{aquino2009generalized}.  
The set $\lbrace\psi_i(r)\rbrace_{i=0}^{n-1}$ is the set of finite element basis functions associated with each node in the mesh 
and $\xi_\alpha(r)$ are enrichment functions.
$\lbrace\psi_i(r)\rbrace_{i=0}^{n-1}$ forms a partition of unity.
$u_i$ and $a_{j\alpha}$ are the classical and enriched nodal coefficients.

The enrichment functions may be anything \JT{give some examples}.
In Ref \cite{aquino2009generalized} and \ref{shilt2018data} global POD modes are used as enrichment functions.
In Ref \cite{babuska2011optimal} local POD modes are used as part of a multi-scale GFEM framework.

\subsection{Connection between GFEM and OD Galerkin}

The FOM is assumed to be discretized using a standard finite element approach (no enrichment functions) such that a function $u(r)$ is approximated as 
\begin{equation}
\label{eq:fem}
u(r) \approx \sum_{i=1}^\FomDim \psi_i(r) u_i,
\end{equation}
where $\psi_i(r)$ is the finite element basis function of node $i$.
Making use of our learned POD basis, the solution is approximated as
\begin{equation}
\state(r) \approx \sum_{i=1}^\FomDim \sum_{j=1}^\RomDim \psi_i(r) \basis_{ij}\RedState_j
\end{equation}
for the ROM (either Galerkin or OD Galerkin).
%and the reference solution is approximated as
%\begin{equation}
%\refstate(r) \approx \sum_{i=1}^\FomDim \psi_i(r) \state_{ref,i}
%\end{equation}

This approximation may also be viewed as coarser GFEM approximation with patches defined by the subdomains introduced in section \ref{sec:od_galerkin}.
The subdomains defined in section \ref{sec:od_galerkin} represent an open covering of the domain $\Omega$.
In keeping with the notation of section \ref{sec:patches}, we will refer to the $i^{th}$ subdomain as $\omega_i$.

We define a partion of unity $\lbrace\gamma_i(r)\rbrace_{i=0}^{n-1}$ with respect to $\lbrace\omega_i\rbrace_{i=0}^{n-1}$
to be the piecewise constant function
\begin{equation}
\gamma_i(r) = 
\begin{cases}
1,& \text{if } r\in\omega_i \\
0,& \text{else}
\end{cases}. 
\end{equation}
We additionally define a set of enrichment functions $\lbrace\xi_k(r)\rbrace_{k=1}^\RomDim$ with
\begin{equation}
\xi_k(r) = \sum_{i=1}^\FomDim \psi_i(r) \basis_{ik}.
\end{equation}

We can then write the approximate solution as
\begin{equation}
\state(r) \approx \sum_{i=0}^{n-1} \gamma_i(r) u_i + \sum_{i=0}^{n-1} \sum_{k=1}^\RomDim \gamma_i(r) \xi_k(r)\RedState_{i,k}.
\end{equation}
Therefore, we can express the OD Galerkin method as a special case of the GFEM method with $u_i=0$.

\JT{It's maybe possible to massage things to include a reference state offset, but it gets kinda messy. If the reference state is constant, we could easily include it easily and have $u_i=\refstate$. If it's a function though it creates trouble fitting it into the first term while still satisfying the partition of unity definition.}
